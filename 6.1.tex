\documentclass[11pt,twoside,onecolumn,a4paper,notitlepage]{article}
%\usepackage{geometry}\geometry{left=2.5cm,right=2.5cm,top=2.5cm,bottom=2.5cm}
%\pagestyle{headings}
\usepackage{enumerate} \linespread{1}
\hyphenation{}
\usepackage{fancyhdr}
\pagestyle{fancy}
\fancyfoot{}
\fancyhead[LE,RO]{\textsc \leftmark~~~~\thepage}
\fancyhead[LO,RE]{\textsc \leftmark~~~~\thepage}
\renewcommand{\headrulewidth}{0.4pt}

\begin{document}
\author{writecoffee}
\title{6.1}
\maketitle

{\setlength{\baselineskip}{1\baselineskip}
\setlength{\parindent}{0pt}
\setlength{\parskip}{2ex plus 0.5ex minus 0.2ex}
\begin{enumerate}[1.]
\item
	Think of a few operations can lead to 8, i.e., 4 * 2 = 8, 16 / 2 = 8, 4 + 4 = 8, %
	and my intuition helps me rule out 2nd and 3rd one, so we can just %
	focus on the number 4 and 2 regardless of their positions.
\item
	Notice that the 1st and 2nd operands of the expected expression easily added to 4
\item
 	Also notice that the 3rd and 4th operands could generate 2 by via 6 / 2. Since %
	operator * and / have the same precedence, we could just flip ``/ 2'' and ``* 6''.
\item
	Here come the solution: (3 + 1) / 2 * 6 = 8.
\end{enumerate}
\par}
\end{document}
